% Training and Loss Functions

\subsection{Classification}

The classifier is a 3-layer MLP that predicts real/fake labels:

\begin{equation}
\begin{aligned}
\mathbf{h}_1 &= \text{ReLU}(\mathbf{z}\mathbf{W}_{c1} + \mathbf{b}_{c1}) \in \mathbb{R}^{256} \\
\mathbf{h}_2 &= \text{ReLU}(\mathbf{h}_1\mathbf{W}_{c2} + \mathbf{b}_{c2}) \in \mathbb{R}^{64} \\
\hat{y} &= \sigma(\mathbf{h}_2\mathbf{W}_{c3} + \mathbf{b}_{c3}) \in [0, 1]
\end{aligned}
\end{equation}

where $\sigma$ is the sigmoid function. Dropout with rate 0.3 is applied after each layer.

\subsection{Loss Functions}

\subsubsection{Classification Loss}
Binary cross-entropy loss for real/fake classification:

\begin{equation}
\mathcal{L}_{\text{cls}} = -\frac{1}{N}\sum_{i=1}^{N} \left[y_i \log \hat{y}_i + (1-y_i) \log (1-\hat{y}_i)\right]
\end{equation}

\subsubsection{Total Loss}
The complete objective combines classification and domain losses:

\begin{equation}
\mathcal{L}_{\text{total}} = \mathcal{L}_{\text{cls}} + \alpha(p) \mathcal{L}_{\text{domain}}
\end{equation}

The gradient reversal ensures that minimizing $\mathcal{L}_{\text{total}}$ simultaneously:
\begin{enumerate}
    \item Improves classification accuracy (via $\mathcal{L}_{\text{cls}}$)
    \item Learns domain-invariant features (via GRL and $\mathcal{L}_{\text{domain}}$)
\end{enumerate}

\subsection{Training Procedure}

\subsubsection{Optimization}
We use AdamW optimizer with:
\begin{itemize}
    \item Learning rate: $\eta = 1 \times 10^{-4}$
    \item Weight decay: $\lambda = 1 \times 10^{-4}$
    \item $\beta_1 = 0.9$, $\beta_2 = 0.999$
\end{itemize}

Learning rate scheduling via cosine annealing:

\begin{equation}
\eta_t = \eta_{\min} + \frac{1}{2}(\eta_{\max} - \eta_{\min})\left(1 + \cos\left(\frac{t\pi}{T}\right)\right)
\end{equation}

where $t$ is the current epoch, $T$ is total epochs, $\eta_{\max}=1\times10^{-4}$, 
and $\eta_{\min}=1\times10^{-6}$.

\subsubsection{Mixed Precision Training}
We employ automatic mixed precision (AMP) with FP16 to reduce memory consumption 
and accelerate training. Gradient scaling prevents underflow:

\begin{equation}
\nabla_{\text{scaled}} = s \cdot \nabla_{\text{fp16}}
\end{equation}

where $s$ is the loss scale factor, dynamically adjusted during training.

\subsubsection{Gradient Accumulation}
To simulate larger batch sizes with limited memory, we accumulate gradients over 
$k=4$ mini-batches before updating:

\begin{equation}
\theta_{t+1} = \theta_t - \eta \frac{1}{k}\sum_{j=1}^{k} \nabla_\theta \mathcal{L}^{(j)}
\end{equation}

Effective batch size: $B_{\text{eff}} = B \times k = 2 \times 4 = 8$.

\subsubsection{Training Algorithm}

\begin{algorithm}[t]
\caption{Multimodal Deepfake Detection Training}
\begin{algorithmic}[1]
\STATE \textbf{Input:} Training set $\mathcal{D}$, epochs $E$, batch size $B$
\STATE \textbf{Initialize:} Encoders, Transformer, GRL, Discriminator, Classifier
\FOR{epoch $e = 1$ to $E$}
    \STATE $p \leftarrow e / E$ \COMMENT{Training progress}
    \STATE $\alpha \leftarrow \alpha_{\text{schedule}}(p)$ \COMMENT{Update GRL alpha}
    \FOR{each batch $\mathcal{B} \subset \mathcal{D}$}
        \STATE Extract tokens: $\{\mathbf{v}_i, \mathbf{a}_i, \mathbf{t}_i, \mathbf{m}_i\}$
        \STATE Add modality embeddings: $\mathbf{H}_0 \leftarrow$ concat + $\mathbf{E}_{\text{mod}}$
        \STATE Apply Transformer: $\mathbf{H}_L \leftarrow \text{Transformer}(\mathbf{H}_0)$
        \STATE Extract CLS: $\mathbf{z} \leftarrow \mathbf{H}_L[0, :]$
        \STATE Classification: $\hat{y} \leftarrow \text{Classifier}(\mathbf{z})$
        \STATE Domain prediction: $\hat{d} \leftarrow \text{DomainDiscrim}(\text{GRL}_\alpha(\mathbf{z}))$
        \STATE Compute $\mathcal{L}_{\text{cls}}$, $\mathcal{L}_{\text{domain}}$
        \STATE $\mathcal{L} \leftarrow \mathcal{L}_{\text{cls}} + \alpha \mathcal{L}_{\text{domain}}$
        \STATE Backpropagate and update parameters
    \ENDFOR
    \STATE Evaluate on validation set
    \STATE Save checkpoint if best accuracy
\ENDFOR
\STATE \textbf{Return:} Best model checkpoint
\end{algorithmic}
\end{algorithm}

\subsection{Implementation Details}

\subsubsection{Hardware}
All experiments conducted on NVIDIA RTX A6000 GPU (48GB VRAM) with:
\begin{itemize}
    \item CUDA 11.8
    \item PyTorch 2.0.1
    \item Mixed precision training enabled
\end{itemize}

\subsubsection{Hyperparameters}
\begin{table}[h]
\centering
\caption{Hyperparameter Configuration}
\label{tab:hyperparams}
\begin{tabular}{lc}
\toprule
\textbf{Parameter} & \textbf{Value} \\
\midrule
Model dimension ($d$) & 512 \\
Transformer layers ($L$) & 4 \\
Attention heads ($h$) & 8 \\
FFN dimension & 2048 \\
Dropout rate & 0.3 \\
Batch size ($B$) & 2 \\
Gradient accumulation ($k$) & 4 \\
Learning rate ($\eta$) & $1 \times 10^{-4}$ \\
Weight decay ($\lambda$) & $1 \times 10^{-4}$ \\
Epochs ($E$) & 10 \\
Max GRL alpha ($\alpha_{\max}$) & 0.5 \\
\bottomrule
\end{tabular}
\end{table}

\subsubsection{Data Preprocessing}
\begin{itemize}
    \item \textbf{Images}: Resized to $224 \times 224$, normalized with ImageNet statistics
    \item \textbf{Audio}: Resampled to 16kHz, segmented into 10s chunks
    \item \textbf{Video}: Sample $F=5$ uniformly spaced frames per clip
    \item \textbf{Text}: Truncated/padded to 256 tokens maximum
\end{itemize}

\subsubsection{Data Augmentation}
Training augmentation includes:
\begin{itemize}
    \item Random horizontal flip (p=0.5)
    \item Random rotation ($\pm 10°$)
    \item Color jitter (brightness, contrast, saturation $\pm 20\%$)
    \item Temporal jittering for video frames
\end{itemize}

\subsubsection{Freezing Strategy}
To reduce computational cost and leverage pretrained knowledge:
\begin{itemize}
    \item Freeze ViT backbone (only train projection layer)
    \item Freeze Wav2Vec2 backbone (only train projection)
    \item Freeze Sentence-BERT (only train projection)
    \item Train Transformer, GRL, Discriminator, Classifier from scratch
\end{itemize}

This reduces trainable parameters from 500M+ to ~50M, enabling efficient training 
while maintaining strong performance.
